\documentclass[12pt,a4paper]{article}
\usepackage[utf8]{inputenc}
\usepackage[francais]{babel}
\usepackage[colorlinks=true,linkcolor=black,linktoc=all]{hyperref}
\usepackage[linesnumbered, ruled, french,onelanguage]{algorithm2e}
\usepackage{graphicx}
\usepackage[left=2cm,right=2cm,top=2cm,bottom=2cm]{geometry}
\usepackage{colortbl}
\setlength{\parindent}{0pt}
\newcommand\tab[1][0.65cm]{\hspace*{#1}}

\begin{document}
\begin{titlepage}
	\begin{figure}
	\centering
	\begin{minipage}{.5\textwidth}
  	\includegraphics[width=.5\linewidth]{ISIR.png}
	\end{minipage}%
	\begin{minipage}{.5\textwidth}
  	\centering
  	\includegraphics[width=.8\linewidth]{SU.jpg}
	\end{minipage}
	\end{figure}
	\centering
	\par
	{\scshape\large \  \par}
	\vspace{2.5cm}
	{\huge\bfseries Rapport de stage :\\
		Design, Réalisation, et Évaluation des techniques d'interaction\par}
	\vspace{2cm}
	{\Large\ B. Thanh Luong \par}
	\vspace{0.5cm}
	{$1^{\texttt{\footnotesize ère}}$ année Spécialité ANDROIDE\\}
	{Année universitaire 2017 - 2018}
	\vfill
	\begin{minipage}{.5\textwidth}
	\centering
  	Responsables pédagogiques :\par
	Pierre Fouilhoux : \href{mailto:pierre.fouilhoux@lip6.fr}{pierre.fouilhoux@lip6.fr}\par
	Viet Hung Nguyen : \href{mailto:hung.nguyen@lip6.fr}{hung.nguyen@lip6.fr}\par
	\end{minipage}%
	\begin{minipage}{.5\textwidth}
  	\centering
  	Tuteurs :\par
	Gilles Bailly : \href{mailto:gilles.bailly@upmc.fr}{gilles.bailly@upmc.fr}\par
	Sylvain Malacria : \href{mailto:sylvain.malacria@inria.fr}{sylvain.malacria@inria.fr}
	\end{minipage}
	\vfill

% Bottom of the page
	{\large août 2018\par}
\end{titlepage}
{\huge \textbf{Remerciements}}\\
\vspace{0.5cm}\\
Je tiens à remercier tout le personnel de l'ISIR pour son accueil
chaleureux ainsi que toutes les personnes qui ont contribué au succès de tout au long de mon stage et les diverses connaissances qu'elles ont partagé avec moi durant toute cette période.
\vspace{0.5cm}\\
Tout d'abord, j'adresse mes remerciements à tous les membres du groupe HCI de l'équipe Interaction, à savoir mes tuteurs M. Gilles Bailly et M. Sylvain Malacria pour leur disponibilité et leurs conseils.
\vspace{0.5cm}\\
Je tiens ensuite à remercier M. Cédric Honnet et M. Marc Teyssier qui ont également été très disponible pour la réparation des matériels.
\vspace{0.5cm}\\
Enfin, je tiens à remercier les responsables du Master ANDROÏDE qui m'ont permis d'effectuer ce stage afin de compléter ma formation d'ingénieur avec leurs cours. Ces connaissances complémentaires m'ont permis d'être encore plus performant lors de mon stage en entreprise et de trouver des
solutions auxquelles je n'aurais peut être pas pensé auparavant.
\newpage
{\huge \textbf{Résumé}}\\
\vspace{0.5cm}\\
Dans ce rapport de stage sont présentées les différentes étapes de développement des techniques d'interaction et de leur évaluation via une application d'édition de texte. Cette application a pour but de
permettre aux chercheur du domaine interaction d'étudier des comportements d'utilisateurs sur les techniques différents. L'application fournit 4 techniques d'utilisation des raccourcis dont une est déjà implémentée. Les techniques d'interaction permettent aux utilisateurs d'avoir une nouvelle perception du lien entre les commandes et leur raccourcis. Le but ici est d'avoir une vue d'ensemble sur ces techniques et de déduire la meilleure parmi celles proposées. Parmi les 4 techniques, deux techniques sont visuelles et deux sont physiques.

Ce stage se déroule du 6 juin au 31 juillet 2018 à l'ISIR qui est un laboratoire de recherche commune à Sorbonne Université et au Centre National de la Recherche Scientifique (CNRS).
\newpage
\tableofcontents
\newpage
\section{Introduction}
Dans le cadre de ma première année de Master Informatique spécialité ANDROÏDE à Sorbonne Université, je souhaite effectuer un stage d'été d'une durée de 2 mois. Il me permet d'être formé au sein d'un laboratoire dans le but d'acquérir des connaissances sur un secteur d'activité, tout en me permettant de mettre en pratique les connaissances théoriques que j'ai acquises lors de mon cursus.

Dans ce rapport, je présente mon environnement de travail ainsi que la mission principale que j'ai réalisée au sein du laboratoire ISIR, à savoir le développement des techniques d'interaction et leur évaluation.
\section{Environnement}
\subsection{Le laboratoire}
\begin{center}
	\includegraphics[width=.3\linewidth]{ISIR.png}
\end{center}
L'Institut des Systèmes Intelligents et de Robotique (ISIR), qui a été créé au $1^{\texttt{\footnotesize er}}$ janvier 2007, est un laboratoire de recherche pluridisciplinaire qui rassemble des chercheurs et enseignants-chercheurs relevant de différentes disciplines des Sciences de l’Ingénieur et de l’Information ainsi que des Sciences du Vivant.

L’ISIR est une Unité Mixte de Recherche (UMR7222) commune à Sorbonne Université et au Centre National de la Recherche Scientifique (CNRS). L'ISIR est rattaché d’une part à la faculté d’Ingénierie de Sorbonne Université (UFR 919) et d’autre part à l’Institut des Sciences de l'Information et de leurs Interactions (INS2I) du CNRS. L’Institut national de la santé et de la recherche médicale (INSERM) est également tutelle de l'une de ses équipes, l’Équipe de recherche labellisée (ERL) U1150.

Les recherches menées à l'ISIR portent sur la modélisation, l'analyse et la conception de systèmes dynamiques et de systèmes de perception. L'ISIR développe des travaux de recherche de haut niveau en s'appuyant sur des équipes pluridisciplinaires regroupant des spécialistes de divers domaines scientifiques des sciences de l'ingénieur et des sciences et techniques de l'information et des neurosciences. Les projets sont organisés au sein de quatre équipes regroupant les personnels autour d'objectifs cohérents, tant du point de vue des finalités que de celui des méthodes développées :  l'équipe Assistance aux Gestes et Applications THErapeutiques (AGATHE), l'équipe Architectures et Modèles pour l'Adaptation et la Cognition (AMAC), l'équipe Interaction, et l'équipe SYstèmes RObotiques COmplexes (SYROCO).

L'organisation du laboratoire :\\
\begin{center}
	\includegraphics[width=1\linewidth]{"Organigramme ISIR - 2018-04".jpg}
\end{center}
\subsection{L'encadrement}
\end{document}