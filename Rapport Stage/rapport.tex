\documentclass[12pt,a4paper]{article}
\usepackage[utf8]{inputenc}
\usepackage[francais]{babel}
\usepackage[colorlinks=true,linkcolor=black,linktoc=all]{hyperref}
\usepackage[linesnumbered, ruled, french,onelanguage]{algorithm2e}
\usepackage{graphicx}
\usepackage[left=2cm,right=2cm,top=2cm,bottom=2cm]{geometry}
\usepackage{colortbl}
\setlength{\parindent}{0pt}
\newcommand\tab[1][0.65cm]{\hspace*{#1}}

\begin{document}
\begin{titlepage}
	\begin{figure}
	\centering
	\begin{minipage}{.5\textwidth}
  	\includegraphics[width=.5\linewidth]{ISIR.png}
	\end{minipage}%
	\begin{minipage}{.5\textwidth}
  	\centering
  	\includegraphics[width=.8\linewidth]{SU.jpg}
	\end{minipage}
	\end{figure}
	\centering
	\par
	{\scshape\large \  \par}
	\vspace{2.5cm}
	{\huge\bfseries Rapport de stage :\\
		Design, Réalisation, et Évaluation des techniques d'interaction\par}
	\vspace{2cm}
	{\Large\ B. Thanh Luong \par}
	\vspace{0.5cm}
	{$1^{\texttt{\footnotesize ère}}$ année Spécialité ANDROIDE\\}
	{Année universitaire 2017 - 2018}
	\vfill
	\begin{minipage}{.5\textwidth}
	\centering
  	Responsables pédagogiques :\par
	Pierre Fouilhoux : \href{mailto:pierre.fouilhoux@lip6.fr}{pierre.fouilhoux@lip6.fr}\par
	Viet Hung Nguyen : \href{mailto:hung.nguyen@lip6.fr}{hung.nguyen@lip6.fr}\par
	\end{minipage}%
	\begin{minipage}{.5\textwidth}
  	\centering
  	Encadrants :\par
	Gilles Bailly : \href{mailto:gilles.bailly@upmc.fr}{gilles.bailly@upmc.fr}\par
	Sylvain Malacria : \href{mailto:sylvain.malacria@inria.fr}{sylvain.malacria@inria.fr}
	\end{minipage}
	\vfill

% Bottom of the page
	{\large août 2018\par}
\end{titlepage}
{\huge \textbf{Remerciements}}
\newpage
\tableofcontents
\newpage
\section{Introduction}
Dans le cadre de ma première année de Master Informatique spécialité ANDROÏDE à Sorbonne Université, je souhaite effectuer un stage d'été d'une durée de 2 mois. Il me permet d'être formé au sein d'un laboratoire dans le but d'acquérir des connaissances sur un secteur d'activité, tout en me permettant de mettre en pratique les connaissances théoriques que j'ai acquises lors de mon cursus.

Dans ce rapport, je présente mon environnement de travail ainsi que la mission principale que j'ai réalisée au sein du laboratoire ISIR, à savoir le développement des techniques d'interaction et l'évaluation de ces techniques. En effet, Veolia possède déjà un outil qui lui permet de gérer
ses commandes, mais il s’agit d’une solution provisoire que l’entreprise a décidé de remplacer.
\section{Environnement}
\subsection{Le laboratoire}
\begin{center}
	\includegraphics[width=.3\linewidth]{ISIR.png}
\end{center}
\end{document}